\documentclass{beamer}
%\usetheme{AnnArbor}
%\usetheme{Antibes}
%\usetheme{Bergen}
%\usetheme{Berkeley}
%\usetheme{Berlin}
%\usetheme{Boadilla}
%\usetheme{boxes}
%\usetheme{CambridgeUS}
\usetheme{Copenhagen}
%\usetheme{Darmstadt}
%\usetheme{default}
%\usetheme{Frankfurt}
%\usetheme{Goettingen}
%\usetheme{Hannover}
%\usetheme{Ilmenau}
%\usetheme{JuanLesPins}
%\usetheme{Luebeck}
%\usetheme{Madrid}
%\usetheme{Malmoe}
%\usetheme{Marburg}
%\usetheme{Montpellier}
%\usetheme{PaloAlto}
%\usetheme{Pittsburgh}
%\usetheme{Rochester}
%\usetheme{Singapore}
%\usetheme{Szeged}
%\usetheme{Warsaw}

\title{Introduction to Python}
\usepackage{listings}
\subtitle{Lists and Dictionaries}
\date{29th September 2016}
% - Give the names in the same order as the appear in the paper.
% - Use the \inst{?} command only if the authors have different
%   affiliation.

% Let's get started
\begin{document}

\lstset{language=Python}

\begin{frame}
  \titlepage
\end{frame}

% Section and subsections will appear in the presentation overview
% and table of contents.
\section{Lists}

\subsection{Basic List Syntax}
\begin{frame}[fragile]{Creating a list}
  \begin{itemize}
  \item {
  Lists are denoted by square brackets. Here is an empty list.
  \begin{block}{}
  \begin{lstlisting}{frame=single}
  x = []
  \end{lstlisting}
\end{block}
 } 
 \pause
 \item {
We can also initialise the list with some elements	  
 \begin{block}{}
 \begin{lstlisting}{frame=single}
  x = [1,"hi",72]
  \end{lstlisting}
\end{block}
\pause

  To access these we use the square bracket accessors:
  \begin{block}{}
  \begin{lstlisting}{frame=single}
   print x[0] # prints 1
   print x[1] # prints "hi"
  \end{lstlisting}
\end{block}
  }
  \end{itemize}
  

\end{frame}

\begin{frame}[fragile]{Mutating lists}
  \begin{itemize}
  \item {
  Square brackets also let us change entries in a list.
  \begin{block}{}
  \begin{lstlisting}{frame=single}
  x = [1,"hi",72]
  x[0] = "foo"
  print x # prints ["foo", "hi",72]
  \end{lstlisting}
\end{block}
  }
  \pause
  \item {
  If we want to add something to a list we can use "append" or list concatenation with "+"
  \begin{block}{}
\begin{lstlisting}{frame=single}
  x.append(0.3) # x = x + [0.3] also works 
  print x[-1] # prints 0.3
  \end{lstlisting}
\end{block}
  }
  \end{itemize}
\end{frame}
\begin{frame}[fragile]{Iterating through lists}
  \begin{itemize}
  \item {
  You have already seen looping through lists!
  \begin{block}{}
  \begin{lstlisting}{frame=single}
  for i in range(1,10): #range returns a list
	      print i
  \end{lstlisting}
\end{block}
  }
  \pause
  \item {
  The same syntax works for our own lists.
  \begin{block}{}
\begin{lstlisting}{frame=single}
  x = [1, "hi", 3]
  for i in x:
		    print i
  \end{lstlisting}
\end{block}
  }
  \end{itemize}
\end{frame}
\begin{frame}[fragile]{Iterating through lists}
  You can also loop by index - only do it if you need the index.
  \begin{block}{}
  \begin{lstlisting}{frame=single}
  x = [1,2,3,4]
  for i in range(len(x)): #range returns a list
	      print i, x[i]
  \end{lstlisting}
\end{block}
\end{frame}

\begin{frame}[fragile]{Strings as lists}
  \begin{itemize}
  \item {
  When it comes to square brackets accessing you can treat strings like lists.
  \begin{block}{}
  \begin{lstlisting}{frame=single}
    x = "hello there"
    print x[1] #prints e
    for char in x: #iterates by character
            print char 
  \end{lstlisting}
\end{block}
  }
  \pause
  \item {
  But do not try to change individual characters in a string.
  \begin{block}{}
\begin{lstlisting}{frame=single}
  x[1] = 'i' #Error
  \end{lstlisting}
\end{block}
  }
  \end{itemize}
\end{frame}

\begin{frame}
	Example Time!
\end{frame}

\subsection{Pythonic Syntax - Slicing}
\begin{frame}[fragile]{Slicing}
  \begin{itemize}
  \item {
  In built into Python are slices - ways of taking sublists of lists and treating them as lists!
  
  Remember, closed left hand - open right hand intervals are the norm.
  \begin{block}{}
  \begin{lstlisting}{frame=single}
  x = [1,"hi",72,0.3]
  print x[1:3] # prints ["hi", 72]
  \end{lstlisting}
\end{block}
  \pause
  \begin{block}{}
  \begin{lstlisting}{frame=single}
  print x[:2] # prints [1, "hi"]  = x[0:2]
  \end{lstlisting}
\end{block}
  \pause
  \begin{block}{}
  \begin{lstlisting}{frame=single}
  print x[2:] #prints [72, 0.3] = x[2:4]
  \end{lstlisting}
\end{block}
  \pause
  \begin{lstlisting}{frame=single}
  print x[:] # Any guesses?
  \end{lstlisting}
  }
  \end{itemize}
  

\end{frame}

\begin{frame}[fragile]{Slicing with steps}
  \begin{itemize}
  \item {
  Using a second colon, we can specify the step size to move through a list.
  \begin{block}{}
  \begin{lstlisting}{frame=single}
  x = [1,"hi",72,0.3]
  print x[:3:2] # prints [1, 72]
  \end{lstlisting}
\end{block}
  \pause
  \begin{block}{}
  \begin{lstlisting}{frame=single}
  print x[:2:3] # prints [1] 
  \end{lstlisting}
\end{block}
  \pause
  \begin{block}{}
  \begin{lstlisting}{frame=single}
  print x[::-1] #prints [0.3,72, "hi", 1] 
  \end{lstlisting}
\end{block}
  }
  \end{itemize}
  

\end{frame}
\begin{frame}
	Example Time!
\end{frame}

\section{Dictionaries}
\subsection{Using Dictionaries}
\begin{frame}[fragile]{Creating a dictionary}
  \begin{itemize}
\item{
Dictionaries are a way of storing a mapping from some data to other data. 
}
  \item {
  Dictionaries are denoted by squiggly brackets. 
  \begin{block}{}
  \begin{lstlisting}{frame=single}
  x ={} 
  \end{lstlisting}
\end{block}
 } 
 \pause
 \item {
We can also initialise the dictionary with some elements	  
 \begin{block}{}
 \begin{lstlisting}{frame=single}
  x = {1 : "one", 2: "two", 3 : "three"  }
  \end{lstlisting}
\end{block}
\pause

  To access these we use the square bracket accessors:
  \begin{block}{}
  \begin{lstlisting}{frame=single}
   print x[1] # prints "one" 
   print x[3] # prints "three" 
  \end{lstlisting}
\end{block}
  }
  \end{itemize}
  

\end{frame}
\begin{frame}[fragile]{Example dictionary}
\begin{itemize}
 \item {But we don't need to use integers as our index set.
 \begin{block}{}
 \begin{lstlisting}{frame=single}
  x = {"four" : 4, "five": 5, "six" : 6  }
  print x["four"] # prints 4
  \end{lstlisting}
\end{block}
}
\pause
\item {We can also add new mappings to our dictionary
 \begin{block}{}
 \begin{lstlisting}{frame=single}
  x["seven"] = 7
  x["six"] = 11
  \end{lstlisting}
\end{block}
}
 \end{itemize}
  

\end{frame}
\begin{frame}
	Example Time!
\end{frame}

\end{document}


