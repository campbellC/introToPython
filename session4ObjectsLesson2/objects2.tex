\documentclass{beamer}
\usetheme{Copenhagen}

\title{Introduction to Python}
\usepackage{listings}
\subtitle{Objects 2: Inheritance and Hooks}
\date{}
\newcommand{\snippet}{\lstinline}
\begin{document}
\definecolor{keywords}{RGB}{255,0,90}
\definecolor{comments}{RGB}{0,0,113}
\definecolor{red}{RGB}{160,0,0}
\definecolor{green}{RGB}{0,150,0}
\lstset{language=Python, 
        basicstyle=\ttfamily\small, 
        keywordstyle=\color{keywords},
        commentstyle=\color{comments},
        stringstyle=\color{red},
        showstringspaces=false,
        identifierstyle=\color{green},
        procnamekeys={def,class}}
\begin{frame}
  \titlepage
\end{frame}

\section{Inheritance}

\begin{frame}[fragile]{Why Objects?}
 	Object-oriented programming can be used for:
 	\begin{itemize}
 		\item Encapsulation
 		\item Abstraction
 		\item \textbf{Code reuse}
 	\end{itemize}
\end{frame}

\begin{frame}[fragile]{Inheritance}
	\begin{itemize}
	\item { Inheritance is a mechanism by which one class \textbf{inherits} code from another class.
	}
	\pause
	\item {Best seen by example. Lets write a program to model cars.
  \begin{block}{}
\begin{lstlisting}{frame=single}
class Car():
    def changeGear(self, gear):
        self.gear = gear 
	
    def printMe(self):
        print "I'm a car."	    
\end{lstlisting}
\end{block}
	}
	\end{itemize}
\end{frame}
\begin{frame}[fragile]{Inheritance}
  \begin{block}{}
\begin{lstlisting}{frame=single}
class Car():
    def changeGear(self, gear):
        self.gear = gear 
	
    def printMe(self):
        print "I'm a car."	    

class Ford(Car):
    def printMe(self):
        print "I'm a Ford!"
x = Ford()
x.printMe() # prints "I'm a Ford!"
x.changeGear(5)
print x.gear #prints 5
\end{lstlisting}
\end{block}

\end{frame}
\begin{frame}[fragile]{How does inheritance work?}
	\begin{itemize}
	\item {Every class inherits from \snippet{object} implicitly.  }
	\pause
	\item {If \snippet{X} inherits from \snippet{Y} then we say \snippet{X} is a \textbf{subclass} of \snippet{Y}.}
	\pause 
	\item {When you call a method on an instance the interpreter checks if the class has that method. If not it looks up the inheritance chain until it finds it.}
	\end{itemize}
\end{frame}
\begin{frame}[fragile]{Constructors of Subclasses}
If you need to initialise data then you must tell your superclass to initialise itself as well.
  \begin{block}{}
\begin{lstlisting}{frame=single}
class Dog():
    def __init__(self, colour):
        self.colour = colour 

class BlackDog(Dog):
    def __init__(self, name):
        Dog.__init__(self, "Black")
        self.name = name
\end{lstlisting}
\end{block}
Make sure to initialise everything in the superclass before setting up the subclass.	
\end{frame}
\section{Hooks}
\begin{frame}[fragile]{What is a hook?}
	\begin{itemize}
	\item { Consider the following code:
  \begin{block}{}
\begin{lstlisting}{frame=single}
print 1+2 #  3 
print "h" + "i" # hi
print ["h"] + ["i"] # ["h","i"]
\end{lstlisting}
\end{block}
	}
	\pause
	\item {All of these types (\snippet{int}, \snippet{str}, \snippet{list}) know how to respond 
	to \snippet{print} and \snippet{+}.}
	\pause
	\item { Using \snippet{print} on our classes so far has resulted in fairly useless output.}
	\end{itemize}
\end{frame}
\begin{frame}[fragile]{What is a hook?}
	\begin{itemize}
	\item { Recall the \snippet{Rational} class from last time:
  \begin{block}{}
\begin{lstlisting}{frame=single}
class Rational():
    def __init__(self, numerator, denominator):
        self.numerator = numerator
        self.denominator = denominator

half = Rational(1,2)
print half 
\end{lstlisting}
\end{block}
	Prints "\_\_main\_\_.Rational instance at 0x10a183440"
	}
	\pause
	\item {Because I haven't told \snippet{Rational} how to respond to print, it defaults to printing 
	its location in memory.}
	\pause
	\item {Bonus question: how does it know how to do that?}
	\end{itemize}
\end{frame}

\begin{frame}[fragile]{What is a hook?}
	\begin{itemize}
	\item { Let's fix that behaviour:
  \begin{block}{}
\begin{lstlisting}{frame=single}
class Rational():
    def __init__(self, numerator, denominator):
        self.numerator = numerator
        self.denominator = denominator

    def __repr__(self):
        return str(self.numerator) + '/' + str(self.denominator)
\end{lstlisting}
\end{block}
	}
	\pause
	\item { \snippet{__repr__} is a \textbf{hook} that the statement \snippet{print} calls on 
	it's arguments.}
	\pause
	\item {This is very useful for keeping track of what your program is doing!}
	\item {\snippet{__} is pronounced \textbf{dunder}- so dunder repr}
	\end{itemize}
\end{frame}

\begin{frame}[fragile]{Important Hooks}
	There are a ton of hooks that you can decide whether to implement for your class. Here are a few I find useful (see Python documentation for full list):
	\begin{itemize}
	\item {\snippet{__eq__(self, other)}: How to respond to comparison with \snippet{==}.}
	\item {\snippet{__add__(self,other)}: How to respond to \snippet{+}}.
	\item{\snippet{__getitem__(self,other)}: implement access to data using the \snippet{x[integer]} notation.}
	\end{itemize}
	Use wisely. If you don't think hard about this you can really confuse your users (or yourself)!
\end{frame}

\subsection{Demo}
\begin{frame}[fragile]{Demo}
	 \huge{\textbf{Demo Time!}}
\end{frame}

\end{document}



