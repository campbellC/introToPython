\documentclass{beamer}
\usetheme{Copenhagen}

\title{Introduction to Python}
\usepackage{listings}
\subtitle{Scripting in Python}
\date{25th November 2016}

\begin{document}

\lstset{language=Python}

\begin{frame}
  \titlepage
\end{frame}




\section{File I/O}

\begin{frame}[fragile]{Opening Files}
\begin{block}{}
\begin{lstlisting}{frame=single}
f = open('foo.txt','r') # read
f.close()
\end{lstlisting}
\end{block}
\pause
 \begin{block}{}
\begin{lstlisting}{frame=single}
f = open('foo.txt','w') # write
f.close()
\end{lstlisting}
\end{block}
\pause
 \begin{block}{}
\begin{lstlisting}{frame=single}
f = open('foo.txt','a') # append
f.close()
\end{lstlisting}
\end{block}
\end{frame}
\begin{frame}[fragile]{Opening Byte Files}
If your file is not a simple text file you might have to treat it as a byte file. e.g. PDF, pickle files etc
\begin{block}{}
\begin{lstlisting}{frame=single}
f = open('foo.txt','rb') # read
f.close()
\end{lstlisting}
\end{block}
 \begin{block}{}
\begin{lstlisting}{frame=single}
f = open('foo.txt','wb') # write
f.close()
\end{lstlisting}
\end{block}
 \begin{block}{}
\begin{lstlisting}{frame=single}
f = open('foo.txt','ab') # append
f.close()
\end{lstlisting}
\end{block}
\end{frame}


\begin{frame}[fragile]{Better Opening Files}
\begin{itemize}
\item You should always close a file that you open.
\item The OS \emph{should} take care of it if your program closes.
\pause
\item The best way to handle this is with \emph{context handlers}:
  \begin{block}{}
\begin{lstlisting}{frame=single}
with open('foo.txt','r') as f:
    # code
\end{lstlisting}
\end{block}
\end{itemize}
\end{frame}

\begin{frame}[fragile]{Reading Files}
	\begin{itemize}
	\item {The file handle has an internal state - position in the file.
	}
	\pause
	\item {Several of the "read" functions move that state forwards}
	\pause
	\item {Use seek() to move around this state}
	\pause
	\item tell() returns the current line number
\begin{block}{}
\begin{lstlisting}{frame=single}
with open('foo.txt','r') as f:
    lines = f.readlines() # list of lines     
    print f.readlines() # prints []
    f.seek(0)
    f.readlines() == lines # True
\end{lstlisting}
\end{block}
\end{itemize}
\end{frame}


\begin{frame}[fragile]{Reading Files}
	\begin{itemize}
	\item {readlines() returns an array of the lines
	}
	\item {read() returns a string of the whole file }
	\pause
	\item {Both of these require you to store whole file in memory, bad for big files}
	\pause
	\item Iteration to the rescue!
\begin{block}{}
\begin{lstlisting}{frame=single}
with open('foo.txt','r') as f:
    for line in f:
        #analyse line 
    # f is now at end of file, seek to restart
\end{lstlisting}
\end{block}
\end{itemize}
\end{frame}



\begin{frame}[fragile]{Writing Files}
  \begin{block}{}
\begin{lstlisting}{frame=single}
with open('foo.txt','w') as f:
    f.write("Hello there\n") # note the \n
\end{lstlisting}
\end{block}
\pause
\begin{block}{}
\begin{lstlisting}{frame=single}
    x = "foo" + str(100) + "bar \n"
    f.write(x)
\end{lstlisting}
\end{block}
\end{frame}


\subsection{Demo}
\begin{frame}[fragile]{Demo}
	 \huge{\textbf{Demo Time!}}
\end{frame}
\section{Pickling}
\begin{frame}[fragile]{Serialisation}
\begin{itemize}
\item Sometimes you need to save your data (or send it).
\pause
\item We have seen how to write to a file - great for numbers, text etc.
\item What about more complicated objects?
\pause
\item Python's pickle library lets you store your objects to disk
\end{itemize}
\end{frame}
\begin{frame}[fragile]{Pickle example}
\begin{block}{}
\begin{lstlisting}{frame=single}
x = some_huge_data_structure()
# Lets store it on disk
pickle.dump(x, open('mySaveFile.pb','wb')) 
# ... some other script
y = pickle.load(open('mySaveFile.pb', 'rb))
\end{lstlisting}
\end{block}
\end{frame}

\section{Important Libraries}
\subsection{PDF handling}
\begin{frame}[fragile]{Not enough time}
	\begin{itemize}
	\item {PyPDF2 is a comprehensive PDF python library
	}
	\item {Concatenate PDF files, create new ones etc.}
	\pause
	\item Demo time - splitter.py
	\end{itemize}
\end{frame}
\subsection{The OS library}
\begin{frame}[fragile]{os}
	\begin{itemize}
	\item {The os library lets you interact with the file system.  }
	\item e.g. view the files in the current directory, sort by last time modified
	\pause
	\item As always, the docs are the place to go - good detailed info on all the useful functions.
	\end{itemize}
\end{frame}

\begin{frame}[fragile]{os - important methods}
\begin{block}{}
\begin{lstlisting}{frame=single}
os.chdir(path) #change current directory
os.listdir(path) # list files in path
os.rename(src, dest) # rename file
os.execv(path, args) # execute program at path
os.tmpfile() # a temporary file
\end{lstlisting}
\end{block}
\end{frame}

\subsection{Regular Expressions}
\begin{frame}[fragile]{re}
	\begin{itemize}
	\item {Regular expressions (Regex) are one of the most powerful single tools in programming.
	}
	\pause
	\item Spend a few hours learning it, its worth it!
	\item Plenty of resources online, especially http://www.regular-expressions.info/tutorial.html
	\item Python library is called re
	\end{itemize}
\end{frame}



\subsection{Command Line Arguments}
\begin{frame}[fragile]{Command Line Arguments}
	\begin{itemize}
	\item {When you launch a program the OS hands it the arguments it was launched with.
	} 
	\pause
	\item{ e.g.
\begin{block}{}
\begin{lstlisting}{frame=single}
python test.py
\end{lstlisting}
\end{block}}
\pause
\item {Access these using the sys module in Python
\begin{block}{}
\begin{lstlisting}{frame=single}
import sys
sys.argv # list of arguments 
sys.argv[0] # name of current program
\end{lstlisting}
\end{block}}
\end{itemize}
\end{frame}
\begin{frame}[fragile]{The argparse Library}
	\begin{itemize}
	\item {For robust argument handling use the argparse library. } 
	\pause
	\item{ e.g.
\begin{block}{}
\begin{lstlisting}{frame=single}
python test.py -f 20 --input-file demoFile.dat
\end{lstlisting}
\end{block}}
	\pause
	\item {Arguments have long and short versions, can be optional, be entered with different orders etc.} 
\end{itemize}
\end{frame}


\subsection{Demo}
\begin{frame}[fragile]{Demo}
	 \huge{\textbf{Demo Time - todo generator}}
\end{frame}




\end{document}
