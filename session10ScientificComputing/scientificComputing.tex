\documentclass{beamer}
\usetheme{Copenhagen}

\title{Introduction to Python}
\usepackage{listings}
\subtitle{Scientific Computing with Python}
\date{2nd December 2016}

\begin{document}

\lstset{language=Python}

\newcommand{\snippet}{\lstinline}

\begin{frame}
  \titlepage
\end{frame}



\section{Options}


\begin{frame}[fragile]{Choices choices}
\begin{itemize}
\item Python has {\em a lot} of options for scientific computing
\item Today is a small selection.
\pause
\item Today: numpy, matplotlib, (no sympy - sorry)
\pause
\item Missed out: Pandas, SciPy, PyGSL, ScientificPython, GmPy\ldots
\item Please \bf{read the docs!}
\end{itemize}
\end{frame}

\section{numpy}

\begin{frame}[fragile]{Introduction to numpy}
\begin{itemize}
\item What is it?
\begin{block}{}
Numpy is an extension to Python adding support for large, multi-dimensional arrays and matrices, along with 
a large library of high-level mathematical functions to operate on these arrays.
\end{block}
\pause
\item More importantly - numpy is {\bf fast}.
\item numpy is written in C and so is as fast as e.g. matlab.
\pause
\item Designed for ease of use with matlab - similar functions and can accept .mat files.
\end{itemize}
\end{frame}

\subsection{numpy Classes and Functions}
\begin{frame}[fragile]{ndarray}
\begin{itemize}
\item The basic building block is the \snippet{ndarray} type.
\pause
\item Multidimensional array. Can use it just like a vector, matrix or higher dimensional analogues.
\pause
\item Don't use the \snippet{matrix} class as it will give you odd bugs!
\pause
\item Arrays have a \snippet{shape} field that returns a vector of dimensions.
\begin{block}{}
\begin{lstlisting}{frame=single}
import numpy as np
x = np.random.rand(3) # x is 1d array 
print x.shape # 3
\end{lstlisting}
\end{block}
\end{itemize}
\end{frame}

\begin{frame}[fragile]{ndarray}
\begin{itemize}
\item Many similar functions to matlab.
\begin{block}{}
\begin{lstlisting}{frame=single}
import numpy as np
x = np.random.rand(3) # shape is 3
y = np.eye((3,3)) # identity matrix
z = np.zeros((4,4,4)) # note two sets of brackets
\end{lstlisting}
\end{block}
\end{itemize}
\end{frame}

\begin{frame}[fragile]{ndarray}
\begin{itemize}
\item All Python arithmetic operators are {\em elementwise}.
\item i.e. \snippet{*},\snippet{+},\snippet{**} etc.
\pause
\item Use \snippet{numpy.dot} for matrix-type multiplication.
\pause
\item If shapes don't match well then numpy tries to broadcast - read the docs!
\pause
\item You can also change the shape with \snippet{.reshape}.
\pause
\item Demo!
\end{itemize}
\end{frame}



\begin{frame}[fragile]{ndarray}
\begin{itemize}
\item Remember that numpy is still just a Python library.
\pause
\item Everything you know still applies 
\pause
\item e.g. slicing - slightly nicer syntax for numpy
\begin{block}{}
\begin{lstlisting}{frame=single}
import numpy as np
x = np.arange(6).reshape(4,4)
y = x[3,:] # y is 4th row of x
\end{lstlisting}
\end{block}
\end{itemize}
\end{frame}


\begin{frame}[fragile]{Statistics Functionality}
\begin{itemize}
\item numpy is smart - if you find yourself writing a for loop always try and rewrite in terms numpy functions.
\pause
\item Numpy is designed so that everything that works in one-dimension should work in any dimension.
\item A lot of basic stats is built in: mean, standard deviation etc.
\pause
\item Example: linear regression
\end{itemize}
\end{frame}

\section{Matplotlib}
\begin{frame}[fragile]{What is matplotlib}
\begin{itemize}
\item What is it?
\begin{block}{}
matplotlib is a python 2D plotting library which produces publication quality figures in a variety of hardcopy formats and interactive environments across platforms.
\end{block}
\pause
\item Makes things very simple!
\pause
\item Line plots, scatter plots, histograms, contour plots, 3d plots etc.
\pause
\item Integrates well with numpy
\end{itemize}
\end{frame}

\begin{frame}[fragile]{Basic idea}
\begin{itemize}
\item There is always a 'current figure' and 'current plot'. 
\item You build up the plot as you go before using \snippet{.show()} or \snippet{.savefig("filename")}.
\pause
\item Can have subplots, labelled axes etc.
\pause
\item Best seen through example!
\end{itemize}
\end{frame}

\end{document}
