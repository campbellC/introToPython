\documentclass{beamer}
\usetheme{Copenhagen}

\title{Introduction to Python}
\usepackage{listings}
\subtitle{Debugging Python Programs}
\date{4th November 2016}
\usepackage{graphicx}
\begin{document}

\lstset{language=Python}

\begin{frame}
  \titlepage
\end{frame}

\section{Introduction}

\subsection{Introduction}
\begin{frame}[fragile]{What Is Debugging?}
  \begin{itemize}
  \item A bug is a problem with a program - either it doesn't run or it doesn't do what you want it to do.
  \pause
  \item{ Named for Admiral Grace Hopper who ``debugged" a moth in a Mark II in the 1940s. 
  \begin{figure}
  \includegraphics[width=0.8\textwidth,natwidth=110,natheight=142]{moth.jpg}
  \end{figure}}
  \end{itemize}
\end{frame}
\begin{frame}[fragile]{What Is Debugging?}
  \begin{itemize}
  \item A bug is a problem with a program - either it doesn't run or it doesn't do what you want it to do.
  \item{ Named for Admiral Grace Hopper who ``debugged" a moth in a Mark II in the 1940s. }
  \item{ Many different types of bugs - memory, flow control, using variables out of scope etc.}
  \pause
  \item{ An \textbf{error} is behaviour in a program we don't want.}
  \pause
  \item { The aim of debugging is to \textbf{diagnose} the bug that causes the error. }
  \end{itemize}
\end{frame}

\section{Debugging In The Wild}
\begin{frame}[fragile]{The Devil's Guide To Debugging }
Stolen from Steve McConnell's Code Complete
\begin{enumerate}
\item{Scatter output statements everywhere}
\pause
\item{Debug the program into existence}
\pause
\item{Never back up earlier versions}
\pause
\item{Don't bother understanding what the program should do}
\pause
\item{Use the most obvious fix (fix the symptom!)}
\pause
\end{enumerate}
\end{frame}


\begin{frame}[fragile]{Better Debugging}
	\begin{itemize}
	\item Use the scientific method. 
	\pause
	\item Take your time - debugging can take more time than coding!
	\pause
	\item Set up test cases that you can work out by hand - or at least estimate.
	\end{itemize}
\end{frame}
\begin{frame}[fragile]{pdb}
	\begin{itemize}
	\item { pdb is a Python library that lets you step through your program and access variables.
	\begin{block}{}
  \begin{lstlisting}{frame=single}
  	import pdb
  	pdb.set_trace() #start debugging
  \end{lstlisting}
  \end{block}}
  \pause
  \item{ Some commands: 
  \begin{enumerate}
\item h(elp) - print list of commands (USE THIS!)
\item b(reak) [linenum]- set breakpoint here or at linenum
\item cl(ear) [linenum] - delete breakpoint here or at linenum
\item s(tep) - step into function called in this line 
\item c(ontinue) - execute until breakpoint
\item p [exp] - print expression e.g. a variable
  \end{enumerate}
  }
	\end{itemize}
\end{frame}
\subsection{Demo}
\begin{frame}[fragile]{Demo}
\begin{itemize}
\item{Classic problem: Strip HTML tags from a string. 
\begin{block}{}
\begin{lstlisting}
	<b>foo</b>	---> foo
\end{lstlisting}
\end{block}}
\item Solution: Have a boolean TAG variable that checks if you are inside a tag or not. 
\end{itemize}

\pause	 
	 \huge{\textbf{Demo Time!}}
	 
\end{frame}

\end{document}




