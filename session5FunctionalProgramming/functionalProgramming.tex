\documentclass{beamer}
%\usetheme{AnnArbor}
%\usetheme{Antibes}
%\usetheme{Bergen}
%\usetheme{Berkeley}
%\usetheme{Berlin}
%\usetheme{Boadilla}
%\usetheme{boxes}
%\usetheme{CambridgeUS}
\usetheme{Copenhagen}
%\usetheme{Darmstadt}
%\usetheme{default}
%\usetheme{Frankfurt}
%\usetheme{Goettingen}
%\usetheme{Hannover}
%\usetheme{Ilmenau}
%\usetheme{JuanLesPins}
%\usetheme{Luebeck}
%\usetheme{Madrid}
%\usetheme{Malmoe}
%\usetheme{Marburg}
%\usetheme{Montpellier}
%\usetheme{PaloAlto}
%\usetheme{Pittsburgh}
%\usetheme{Rochester}
%\usetheme{Singapore}
%\usetheme{Szeged}
%\usetheme{Warsaw}

\title{Introduction to Python}
\usepackage{listings}
\subtitle{Functional Programming in Python}
\date{\today}

\begin{document}

\lstset{language=Python}

\begin{frame}
  \titlepage
\end{frame}

\section{Introduction}

\subsection{Introduction}
\begin{frame}[fragile]{Lambda Calculus}
  \begin{itemize}
  \item {
  \textbf{Idea:} Create a syntax to describe computation and study it formally as an abstract process.
}
  
  \pause
  \item {
  First formulated by Alonzo Church just as Turing was inventing Turing machines.
 } 
 \pause
 \item {
Formally equivalent to Turing machines (proved by Turing). Led to the Church-Turing Thesis.
  }
  \end{itemize}
\end{frame}

\begin{frame}[fragile]{Lambda Calculus in the Real World}

\begin{itemize}
\item{ \textbf{1958:} Lisp invented - a realisation of the Lambda Calculus. Nowadays: Common Lisp, Scheme, Clojure}
\pause
\item{\textbf{1970s:} ML invented at University of Edinburgh as a language to prove theorems with. Nowadays: OCaml, Standard ML.
}
\pause
\item{\textbf{1987:} Haskell language invented, taught at Edinburgh to first years}
\pause
\item{\textbf{Present day:} Erlang, Python, C++ (ish), Swift and many more!}
\end{itemize}
\end{frame}


\begin{frame}[fragile]{Concepts}
	\begin{itemize}
	\item { \textbf{Functions are First-Class Citizens:} Functions are just examples of data - they can be parameters in functions and return values to functions
	}
	\pause
	\item{ \textbf{Pure functions:} Functions have no side effects. In particular every function is idempotent.
	}
\pause
\item{\textbf{Recursion:} Because of the ``no mutation" philosophy, recursion is always preferred over iteration.
}
\pause
\item{\textbf{Python is \emph{not} a pure-functional language. It's up to you what you use and ignore from the above
}}
\end{itemize}
\end{frame}



\section{Functional Programming In Action}
\subsection{Helper Functions and Closures}
\begin{frame}[fragile]{Helper Functions}
	\begin{itemize}
	\item { You can define functions inside functions. These will be inaccessible to the outside world.
  \begin{block}{}
  \begin{lstlisting}{frame=single}
  def f(n):
  	def g(m):
	return m*m #latex won't let me indent  :(
  	return g(n)
  \end{lstlisting}
\end{block}
	}
\item{ This is useful for de-cluttering your code and hiding functionality you don't want to be public.} 
	\end{itemize}
\end{frame}
\begin{frame}[fragile]{Closures}
	\begin{itemize}
	\item { Helper functions are more powerful than at first they seem. Remember functions can be return values as well!
	}
	\pause
	\item { When you return a function Python will \emph{capture} local variables needed to use later.
  \begin{block}{}
  \begin{lstlisting}{frame=single}
  def powerFactory(n):
  	def g(x):
		return x ** n
  	return g
  \end{lstlisting}
\end{block}
}
\pause
\item{
\large{\textbf{ Demo Time!}}
}
\end{itemize}
\end{frame}
\subsection{Anonymous Functions}
\begin{frame}[fragile]{Lambdas}
	\begin{itemize}
	\item { Sometimes it can be annoying to come up with names for functions you will only use once.
	} 
	\pause
	\item{Use Lambdas to get around this 
  \begin{block}{}
  \begin{lstlisting}{frame=single}
  def powerFactory(n):
  	return (lambda x: x ** n) 
  \end{lstlisting}
\end{block}
}
	\end{itemize}
\end{frame}

\subsection{Demo}
\begin{frame}[fragile]{Demo}
	 \huge{\textbf{Demo Time!}}
\end{frame}



\section{Map, Reduce and Filter}
\begin{frame}[fragile]{Introduction}
	\begin{itemize}
	\item {Map, Reduce and Filter are three common functions from functional programming.
	} 
	\item{Each function acts on a list. Since these are functional they do not mutate the list!
}
\pause
\item{The names have since become famous because of the 'MapReduce' framework invented by Google for Big Data calculations
}
	\end{itemize}
\end{frame}
\begin{frame}[fragile]{Map}
	\begin{itemize}
	\item {Map takes a function of one variable and a list and acts componentwise on the list.
	} 
	\pause
	\item{For example:
  \begin{block}{}
  \begin{lstlisting}{frame=single}
  x = [1,2,3]
  print map(lambda z: z ** 2, x) 
  # prints [1, 4, 9]
  \end{lstlisting}
\end{block}
}
	\end{itemize}
\end{frame}

\begin{frame}[fragile]{Reduce}
	\begin{itemize}
	\item {Reduce takes a function of two variables, a list and an initial value and 'folds' down the list.
	} 
	\pause
	\item{For example:
  \begin{block}{}
  \begin{lstlisting}{frame=single}
  x = [1,2,3]
  print reduce(lambda y,z: y + z, x, 0) 
  # prints 6
  \end{lstlisting}
\end{block}
}
	\end{itemize}
\end{frame}
\begin{frame}[fragile]{Filter}
	\begin{itemize}
	\item {Filter takes a function of one variable that returns True or False, a list and returns the list with only 
	those elements that the function returns True on.
	} 
	\pause
	\item{For example:
  \begin{block}{}
  \begin{lstlisting}{frame=single}
  x = [1,2,3]
  print reduce(lambda z: z \% 2, x) 
  # prints [1,3]
  \end{lstlisting}
\end{block}
}
	\end{itemize}
\end{frame}
\end{document}
