\documentclass{beamer}
\usetheme{Copenhagen}

\title{Introduction to Python}
\usepackage{listings}
\subtitle{The Python Data Model}
\date{10th November 2016}

\begin{document}

\lstset{language=Python}

\begin{frame}
  \titlepage
\end{frame}


\begin{frame}[fragile]{The Function Call Stack}
\begin{itemize}
\item{You start with a `global scope`. }
\item{This is a set of variables, class names etc. available to you at all points in the program.}
\pause
\item{When you call a function you `push` a new scope onto the `call stack`.}
\pause
\item{When a function returns, it is `popped` off the stack.}
\pause
\item{\bf{Demo Time!}}
\end{itemize}
\end{frame}

\begin{frame}[fragile]{How does Python Store Data?}
\begin{itemize}
\item{Recall that everything in Python is an object}
\item{Functions, numbers, strings, python code...}
\pause
\item{Every object has a unique ID number. In CPython (standard Python on Linux) this is the address in memory. (C Pointer if you know what that means) }
\pause
\begin{block}{}
	\begin{lstlisting}
	x = 2
	y = 2
	x == y # True
	x is y # maybe true? don't depend on it!
	\end{lstlisting}
\end{block}
\end{itemize}
\end{frame}

\begin{frame}[fragile]{How does Python Store Data?}
\begin{itemize}
\item{Every object has a reference count - how many things are pointing to me?}
\pause
\item{If you assign a name to an object the reference count goes up. If you remove that assignment it goes down.}
\pause
\item{If the reference count is zero, Python deletes the object for you and frees up the memory.}
\item{This is called \textbf{Garbage Collection}.}
\end{itemize}
\end{frame}



\begin{frame}[fragile]{Mutable vs. Immutable Data Types}
\begin{itemize}
\item{Some objects cannot be changed in place. These are \textbf{Immutable Objects}.}
\begin{block}{}
	\begin{lstlisting}
	x = 2
	y = x 
	x is y #True! guaranteed
	x += 2
	print(y) # prints 2
	\end{lstlisting}
	\end{block}
\pause
\item{Strings, ints and floats are immutable. This makes sense (I think).}
\end{itemize}
\end{frame}


\begin{frame}[fragile]{Mutable vs. Immutable Data Types}
\begin{itemize}
\item{Other objects can be changed in place. These are \textbf{Mutable Objects}.}
\begin{block}{}
	\begin{lstlisting}
	x = [1,2,3] 
	y = x 
	x is y #True! guaranteed
	x.append(2) 
	print(y) # [1,2,3,2]
	\end{lstlisting}
	\end{block}
\pause
\item{Lists and dictionaries are mutable types.}
\pause
\item{ Use round brackets instead of square brackets to create a tuple - an immutable list.}
\end{itemize}
\end{frame}

\begin{frame}[fragile]{Shameless theft}
\begin{itemize}
\item{ http://nedbatchelder.com/text/names1.html} 
\item{I will now steal this guys talk}
\item{This weeks exercises all stolen from reddit.com/r/dailyprogrammer}
\item{Use this site for practice!}
\end{itemize}
\end{frame}


\end{document}




